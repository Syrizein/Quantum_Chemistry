%
% 使用 xelatex 编译
%
\documentclass[aspectratio=169,10pt,compress,t]{ctexbeamer}
\usetheme{Berlin}
\usecolortheme{dolphin}
\usefonttheme[onlymath]{serif}

%%%%% ===== 常用宏包 =======================================================
\usepackage{amsmath,amssymb,amsfonts,bm}
\usepackage[version=3]{mhchem} 
\usepackage{graphicx}
\usepackage{booktabs}
\usepackage{mathrsfs}
\usepackage{braket}
\usepackage[style=verbose]{biblatex}
\usepackage{caption,subcaption}
\usepackage{ulem}
\usepackage{color}
\definecolor{PPTBlue}{RGB}{51,51,178}

\addbibresource{ref.bib}
\setbeamerfont{footnote}{size=\tiny}

\begin{document}

\title[QC1]{Quantum Chemistry I}
\subtitle{Mathematical Backgrounds}

\author[XYJ]{Yunjie Xu}
\institute[HIT]{\zihao{-5} Harbin Institute of Technology}
\date[2022.08]{August 2022}

\begin{frame}[plain]
  \begin{figure}[h]
    \includegraphics[height=3cm]{HIT.png}
    \centering
  \end{figure}
  \titlepage
\end{frame}

\begin{frame}{List of Contents}
  \tableofcontents[hideallsubsections]
\end{frame}

% 在每节前插入目录
\AtBeginSection[]{\frame{\tableofcontents[currentsection,hideallsubsections]}}

\section{Linear Space}
\begin{frame}{Linear Space}
In the very beginning, we have to revise some of the basic notions.

\begin{block}{Linear Space}
    A linear space $V$ on field $\mathcal{F}$ is defined as a set eqquiped with two operations:
    addition $V×V→V$ and scalar product $\mathcal{F}×V→V$, with the operations satisfying:
    \begin{enumerate}
      \item addition associativity: $(u+v)+w=u+(v+w)=u+v+w$
      \item existance of identity element: there exists 0, $0+v=v$
      \item existance of inverse elements: for any $v$ exists $v'$, $v+v'=v'+v=0$
      \item scalar product associativity: $a(bv) = (ab)v = abv$
      \item existance of identity element: there exists 1, $1v=v$
      \item distributivity: $(a+b)v = av+ bv$, $a(u+v) = au+ av$
    \end{enumerate}
\end{block}  
\end{frame}

\begin{frame}{Linear Space}

\begin{block}{pre-Hilbert Space}
    A pre-Hilbert space $\mathcal{F}$ is a linear space eqquiped with inner product 
    $$< , >: \mathcal{F} × \mathcal{F} → \mathbb{C}$$
    satisfying:
    \begin{enumerate}
      \item $<au+bv, w> = a<u,w>+ b<v,w>$
      \item $<u, v> = <v, u>^*$
      \item $<x, x> = \Vert x \Vert^2 \geq 0 $ . Equality holds if and only if $x = 0$.
    \end{enumerate}
\end{block}  

\begin{block}{Hilbert Space}
  A Hilbert space $\mathcal{H}$ is a Cauchy-complete pre-Hilbert space.
\end{block}  
\end{frame}

\begin{frame}{Operators}
\begin{block}{Operators}
  An operator is a mapping $\mathscr{O}: u → v$.
  
  The operator is linear if for any $u, v$:
  $$\mathscr{O}(au+bv)=a\mathscr{O}u+b\mathscr{O}v$$
\end{block}  
\uncover<2->{
Exersice A: Linear operator $\mathscr{O}$ on finite-dimentional linear space $V$ could be represented by
a $dim(V\,) × dim(V\,)$ matrix. Why?
}

\end{frame}

\section{Linear Space in Quantum Mechanics}
\begin{frame}{Dirac Bra-ket Notions}
Dirac denote the vectors in Hilbert spaces as $\ket{a}$.
The corresponding covector is denoted as $\bra{a}$.

The inner product is defined as:
$$\bra{a} \ket{b} = \braket{a|b} = a^\dagger b = \sum a_i^* b_i$$

Tips: use $\mathtt{braket}$ package in \LaTeX.

\bigskip

\uncover<2->{
Exercise B: What does $\ket{v\,}\bra{v\,}$ mean? Show that for a complete set
of basis {$\ket{i\,}$}:
$$\sum \ket{i\,} \bra{i\,} = \mathbf{1}$$
where $\mathbf{1}$ is the identity operator.

}
\end{frame}

\begin{frame}{Functional Notions}
The functional notions corresponds to the wave mechanics.

The vectors are now functions $a(x)$.

The inner product are defined as follows:
$$\int_\Omega dx a^*(x) b(x)$$

Integrals replace summations.
$$\sum \ket{i\,} \bra{i\,} = \mathbf{1}$$
$$\int dx \ket{i\,} \bra{i\,} = \mathbf{1}$$
\end{frame}

\begin{frame}{Operators, Continued}
\begin{block}{Local and Nonlocal Operators}
  An nonlocal operator is defined like:
  $$b(x) = \int_\Omega dx' O(x, x') a(x')$$
  An local operator is defined like:
  $$b(x) = \int_{\delta(x, \epsilon)} dx' O(x, x') a(x')$$
  Example of local operators: $x$, $p$, $L_z$, \dots
\end{block}  
\end{frame}

\begin{frame}{Operators, Continued}
\begin{block}{Commutator and Antiommutator of Operators}
  $$\left[ \mathscr{A}, \mathscr{B} \right] = \mathscr{A} \mathscr{B} - \mathscr{B} \mathscr{A}$$
  $$\left\{ \mathscr{A}, \mathscr{B} \right\} = \mathscr{A} \mathscr{B} + \mathscr{B} \mathscr{A}$$
\end{block}  

\begin{block}{Unitary Matrices and Hermite Matrices}
  Unitary matrices are defined as:
  $A^{-1} = A^{\dagger}$
  
  Hermite matrices are defined as:
  $A^{\dagger} = A$
\end{block}  
\end{frame}

\begin{frame}{Operators, Continued}
\begin{block}{Adjunct (or Hermite transpose)}
  $$A^{\dagger} = (A^T)^* = (A^*)^T$$
\end{block}  

Exercise C: For a Hermite operator(which means the matrix representation of operator is Hermite) $\mathscr{A}$, $\mathscr{A}\ket{a\,} = \omega_a \ket{a\,}$
\begin{enumerate}
  \item Prove that $\omega_a$ is real for any $a$.
  \item Prove that $\braket{a|b\,} = \delta_{ab}$ for any $a, b$
\end{enumerate}

\end{frame}

\section{Change of Basis}

\begin{frame}{Change of Basis Using Dirac Notions}
Assume we have two orthogonal normalised basis {$\ket{i\,}$} and {$\ket{a\,}$}

\begin{equation*}
\begin{aligned}
  \braket{i\,|j\,} = \delta_{ij} \quad & \quad \sum \ket{i\,}\bra{i\,} = \mathbf{1} \\
  \braket{\alpha|\beta} = \delta_{\alpha\beta} \quad & \quad \sum \ket{\alpha}\bra{\alpha} = \mathbf{1}
\end{aligned}
\end{equation*}

To change the basis from {$\ket{i\,}$} to {$\ket{a\,}$}:

\begin{equation*}
\begin{aligned}
  \ket{\alpha} &= \mathbf{1} \ket{\alpha} \\
  &= \sum \ket{i\,}\braket{i\,| \alpha} \\
  &= \sum \ket{i\,}(\mathbf{U})_{i\alpha}
\end{aligned}
\end{equation*}

The matrix $\mathbf{U}$ is called the transformation matrix.
With similar processed we could derive:
$$\ket{i\,} = \sum \ket{i\,}(\mathbf{U}^\dagger)_{\alpha i} $$

\end{frame}

\begin{frame}{Change of Basis Using Dirac Notions}
The matrix representation could be transformed using the same method:

\begin{equation*}
\begin{aligned}
  \mathscr{O} \rightarrow \mathbf{O} \quad & \quad \mathbf{O}_{ij} = \bra{i\,} \mathscr{O} \ket{j\,} \\
  \mathscr{O} \rightarrow \boldsymbol{\Omega} \quad & \quad \boldsymbol{\Omega}_{\alpha \beta} = \bra{\alpha} \mathscr{O} \ket{\beta}
\end{aligned}
\end{equation*}

\begin{equation*}
\begin{aligned}
  \Omega_{\alpha \beta} &= \bra{\alpha} \mathscr{O} \ket{\beta} \\
  &= \bra{\alpha} \mathbf{1} \mathscr{O} \mathbf{1} \ket{\beta} \\
  &= \sum_{ij} \braket{\alpha | i\,} \bra{i\,} \mathscr{O} \ket{j\,} \braket{j\, | \beta} \\
  &= \sum_{ij} (\mathbf{U}^\dagger)_{\alpha i} (\mathbf{O})_{ij} (\mathbf{U})_{j \beta}
\end{aligned}
\end{equation*}

So matrices are connected by unitary transformations:
\begin{equation*}
  \boldsymbol{\Omega} = \mathbf{U}^\dagger \mathbf{OU} \quad \mathbf{O} = \mathbf{U} \boldsymbol{\Omega} \mathbf{U}^\dagger 
\end{equation*}

\end{frame}

\begin{frame}{Diagonalisation}
There are multiple ways of diagonalise an Hermite matrix.
Here we only present the original method.

For a given matrix $\mathbf{O}$, solving the eigenvalue equations:
$$(\mathbf{O}-\lambda \mathbf{I}) \mathbf{c} = 0$$
we obtain $\{ \lambda_i \}$ and $\{ \mathbf{c}^i \}$, and
$$\mathbf{U} = \left( \, \mathbf{c}^1 \; \mathbf{c}^2 \; \dots \; \mathbf{c}^N \, \right)$$
$$\boldsymbol{\Omega} = \mathbf{U}^\dagger \mathbf{OU} =
\begin{pmatrix}
\lambda_1 &  & & \\
 & \lambda_2 & & \\
 & & \ddots & \\
 & & & \lambda_N 
\end{pmatrix}
$$

\end{frame}

\begin{frame}{Function of Matrices}
For diagonalised matrices:
$$
f\,(\mathbf{a}) =
\begin{pmatrix}
f\,(a_1) &  & & \\
 & f\,(a_2) & & \\
 & & \ddots & \\
 & & & f\,(a_N) 
\end{pmatrix}
$$

And for undiagonalised matrices:
$$\mathbf{U}^\dagger \mathbf{AU} = \mathbf{a}$$
$$\mathbf{A} = \mathbf{Ua} \mathbf{U}^\dagger$$
$$f\,(\mathbf{A}) = \mathbf{U} f\,(\mathbf{a}) \mathbf{U}^\dagger $$

\end{frame}

\section{Variational Method}

\begin{frame}{Schrödinger Equation}
\begin{block}{Time-dependent Schrödinger Equation(TDSE)}
  Time-dependent Schrödinger Equation is a eigenvalue equation:
  $$\mathscr{H} \ket{\Phi} = \mathscr{E} \ket{\Phi}$$
  with $\mathscr{H}$ denotes Hamiltonian and $\mathscr{E}$ denotes energy.
\end{block} 
We shall not derive the equation here.

The Hamiltonian of molecular system:
$$\mathscr{H} = -\sum_i \frac{1}{2} \nabla_i^2 -\sum_A \frac{1}{2} \nabla_A^2 - \sum_{i,A} \frac{Z_A}{r_{iA}} + \sum_{i<j} \frac{Z_A}{r_{ij}} + \sum_{A<B} \frac{Z_A Z_B}{R_{AB}}$$
\end{frame}

\begin{frame}{Schrödinger Equation}
There exist a set of accurate solutions to the equation:
$$\mathscr{H} \ket{\Phi_i} = \mathscr{E}_i \ket{\Phi_i}$$

with $\mathscr{E}_0 \leq \mathscr{E}_1 \leq \dots $.

\bigskip

We assume the set is complete, which means any state $\ket{\tilde{\Phi}}$ could be
expressed in the form
$$\ket{\tilde{\Phi}} = \sum c_i \ket{\Phi_i} $$
\end{frame}

\begin{frame}{Variational Principle}
\begin{block}{Variational Principle}
  For a given state $\ket{\Psi}$:

  $$\frac{\Bra{\Psi}\mathscr{H}\Ket{\Psi}}{\Braket{\Psi|\Psi}} \geq \mathscr{E}_0$$

  The equality holds if and only if $\ket{\Psi}$ = $\ket{\Phi_0}$
\end{block} 
\end{frame}

\begin{frame}{Variational Principle}
Now with this result we could write:
$$\ket{\tilde{\Phi}} = \sum c_i \ket{\Psi_i} $$

{$\ket{\Psi_i}$} is a set of known functions.

Define a new matrix $\mathbf{H}$ as:
$$ (\mathbf{H})_{ij} = H_{ij} = \Bra{\Psi_i} \mathscr{H} \Ket{\Psi_i} $$

Assume that the state is normalised:
$$ \braket{\tilde{\Phi} | \tilde{\Phi}} = \sum c_i^2 = 1$$

\end{frame}

\begin{frame}{Variational Principle}

As $c_1, c_2, \dots, c_n$ is not indenpendent, Lagrange multiplier method
must be put in use here:

\begin{equation*}
\begin{aligned}
\mathcal{L}(c_1, \dots, c_N, \lambda)  &= \Bra{\Psi_i} \mathscr{H} \Ket{\Psi_i} - \lambda( \braket{\tilde{\Phi} | \tilde{\Phi}} - 1) \\
&= \sum c_i c_j H_{ij} - \lambda \left( \sum c_i^2 - 1 \right)
\end{aligned}
\end{equation*}

Minimizing $\mathcal{L}$ gives:

$$\frac{\partial \mathcal{L}}{\partial c_k} = \sum c_j H_{kj} + \sum c_i H_{ik} - 2 \lambda c_k = 0$$

$\mathbf{H}$ is symmetric, so:

$$\sum H_{ik} c_k = \lambda c_k$$
$$\mathbf{H} \mathbf{c} = \lambda \mathbf{c}$$

\end{frame}

\begin{frame}{Variational Principle}

The equation
$$\mathbf{H} \mathbf{c} = \lambda \mathbf{c}$$
have $N$ different solutions:
$$\mathbf{H} \mathbf{c^\alpha} = \lambda_\alpha \mathbf{c^\alpha}$$
with $\lambda_0 \leq \dots \leq \lambda_{N-1}$.
As $\mathscr{H}$ is Hermite, $\mathbf{H}$ must be Hermite too, that means:
$$(\mathbf{c}^\alpha)^\dagger \mathbf{c}^\beta = \sum c_i^\alpha c_i^\beta = \delta_{\alpha \beta}$$

\end{frame}

\begin{frame}{Variational Principle}
$$\mathbf{H} \mathbf{c^\alpha} = \lambda_\alpha \mathbf{c^\alpha}$$
If we define two new matrices:
$$(\mathbf{C})_{i\alpha} = C_{i\alpha} = c_i^\alpha$$
$$(\boldsymbol{\Lambda})_{\alpha \beta} = \Lambda_{\alpha \beta} = \lambda_\alpha \delta_{\alpha \beta}$$
The equation is equivalent to:
$$\mathbf{HC}=\mathbf{C}\boldsymbol{\Lambda}$$
\end{frame}

\begin{frame}{Variational Principle}
It could be proved that:
$$\lambda_\alpha = \bra{\tilde{\Phi}_\alpha} \mathscr{H} \ket{\tilde{\Phi}_\alpha} $$
by combining:
$$\ket{\tilde{\Phi}} = \sum c_i \ket{\Psi_i} $$
$$(\mathbf{c}^\alpha)^\dagger \mathbf{c}^\beta = \sum c_i^\alpha c_i^\beta = \delta_{\alpha \beta}$$
$$\mathbf{H} \mathbf{c^\alpha} = \lambda_\alpha \mathbf{c^\alpha}$$

So we would like to denote $\lambda$ as $E$ and re-write the equations as:
$$\mathbf{HC}=\mathbf{CE}$$

\end{frame}

\begin{frame}[c,plain]
\begin{center}
\Huge\color{PPTBlue}\heiti\bfseries Thank you for listening!
\end{center}
\end{frame}

\end{document} 